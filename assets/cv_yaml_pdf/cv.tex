\documentclass[margin]{res}  
% Default font is the helvetica postscript font
\usepackage[utf8]{inputenc}
\usepackage{hyperref}
\usepackage{lmodern}
\usepackage{enumitem}
\addtolength{\topmargin}{-.5in}
\addtolength{\oddsidemargin}{-.4in}
\addtolength{\textwidth}{.6in}
%\usepackage{multibib}
%\newcites{book,misc}{{Books},{Others}}

% Increase text height
\textheight=700pt

\renewcommand{\namefont}{\LARGE\bf\fontfamily{lmss}\selectfont}
\renewcommand{\sectiontitlefont}{\large\fontfamily{lmss}\selectfont}
\renewcommand{\sectionfont}{\fontfamily{lmss}\selectfont}
\newcommand{\contactsfont}{\fontfamily{qcr}\selectfont}


\begin{document}

\name{}

% Note that addresses can be used for other contact information:
% -phone numbers
% -email addresses
% -linked-in profile
\address{
,  \\
\contactsfont{giaga7@gmail.com} \\
\contactsfont{marinig@student.unimelb.edu.au} \\
\contactsfont{(+61) 481149376}
}

\address{
\hfill Melbourne, Australia \\
\hfill \contactsfont{https://gmarini.com}\\
\hfill LinkedIn: \contactsfont{\href{https://www.linkedin.com/in/gabryxx7}{@gabryxx7}} \\
\hfill GitHub: \contactsfont{\href{https://github.com/Gabryxx7}{@gabryxx7}}
}
\begin{resume}

%-------------------------------------------------------------------------------
%	EDUCATION SECTION
%-------------------------------------------------------------------------------
\section{Summary}
{Third year PhD Candidate at the University of Melbourne, sponsored by CSIRO-Data61. My research focuses on indoor localisation, sensors and modeling. I am very passionate about research and especially ubiquitous computing but I often work with 3D graphics (Unity mostly) and Android.}

\section{Education}
\textbf{PhD in Computing and Information Systems} \hfill{May 2018 - Ongoing} \\
\textit{University of Melbourne}. Melbourne, Australia\\
Third year PhD Candidate. Expected graduation: May 2021

\textbf{PhD in Computing and Information Systems} \hfill{May 2018 - Ongoing} \\
\textit{CSIRO - Data61}. Melbourne, Australia

\textbf{MSc in Computer Science (Laurea Magistrale)}\hfill Jun 2017 \\
\textit{Universit\'{a} degli Studi di Pisa}. Pisa, Italy \\
Final grade: 110/110 \textit{cum laude}

\textbf{BSc in Computer Science} \hfill Jul 2014 \\
\textit{Universit\'{a} degli Studi di Cagliari (Laurea Triennale)}. Cagliari, Italy
\\Final grade: 110/110 \textit{cum laude}


%-------------------------------------------------------------------------------
% Modify the format of each position
% \begin{format}
% \title{l}\\
% \dates{l}\location{r}\\
% \body\\
% \end{format}
%-------------------------------------------------------------------------------

\section{Experience}

\textbf{Data Science Research Intern}\hfill{June 2021 - August 2021}\\
\textit{Nokia Bell Labs}. (Remote) Cambridge, UK\\
IoT, sensing and data science internship

\textbf{PhD Candidate}\hfill{Jul 2018 - Ongoing}\\
\textit{University of Melbourne}. Melbourne, Australia\\
\textbullet{} \textbf{Trainees Program Chair} of the \textit{NHMRC Center for Research Excellence}\\
\textbullet{} \textbf{Treasurer} of the CIS Research Students Association (\textit{\textbf{CIS-GReS}})\\
\textbullet{} \textbf{Teaching}: Tutoring includes material creation, teaching and exams/projects marking.
\\\ - INFO20005 (2021): UI Development
\\\ - INFO90003 (2020): Designing Novel Interaction
\\\ - COMP30019 (2018, 2019 and 2020): Computer Graphics and Interaction
\\\ - SWEN20003 (2019): Object Oriented Software Development\\
\textbullet{} \textit{Equipment manager} for the IDL lab\\
\textbullet{} \textit{Floor Warden for the CIS department}

\textbf{Consultant}\hfill{Oct 2017 - Feb 2018}\\
 \textit{European Central Bank (ECB)}. Frankfurt, Germany\\
I worked on the long-term IReF project implementing an online survey to be sent out to all the banks in Europe. Besides that, I offered technical support for IT-related matters (Python scripts, WordPress websites management, Javascript, VBA, SQL)


\textbf{Research Assistant}\hfill{Jun 2017 - Oct 2017}\\
\textit{TeCIP Institute, Scuola Superiore Sant'Anna}. Pisa, Italy\\
I worked mostly on robot vision and Hand-Eye camera robot calibration. I also developed part of the HMI (C\#, MySQL) with a 3D Viewer of the robotic arm with inverse kinematics


\textbf{Intern}\hfill{Oct 2016 - Jun 2017}\\
\textit{VRMedia - Scuola Superiore Sant'Anna}. Pisa, Italy\\
I developed and wrote my master thesis under the supervision of Prof. Franco Tecchia.


%-------------------------------------------------------------------------------
%	COMPUTER SKILLS SECTION
%-------------------------------------------------------------------------------
\section{Technical\\skills}

\textbf{Programming Languages:} C++, C\#, C, PHP, Java, Python
\\
\textbf{Data Analysis/Visualisation:} R, Python, Javascript (and NodeJS), SQL, MySQL, MatLab
\\
\textbf{Web Development:} HTML, CSS, Javascript (and NodeJS), PHP
\\
\textbf{Markup Languages:} Latex, Markdown, YAML, HTML, XML, JSON
\\
\textbf{Tools/Framework:} Many IDEs and notebooks such as PyCharm, Eclipse, Jupyter, RStudio, Unity3D, Photoshop, Blender, WordPress


\section{Talks}
\par
\textbullet{} \textit{Brainstorming Workshop} \hfill{NHMRC CRE 2021}\\
\textbullet{} \textit{Masterclass in Photography for HCI Research} \hfill{University of Melbourne, April 2021}\\
\textbullet{} \textit{Towards Context-Free Semantic Localization} \hfill{UbiComp, London, UK, Sep 2019}\\
\textbullet{} \textit{Towards Indoor Localization Analytics for Modelling Flow of Movements} \hfill{\UbiComp Doctoral Colloquium, London, UK, Sep 2019}\\
\textbullet{} \textit{Flow-Optimisation based on indoor localisation} \hfill{Data61/CSIRO, Aug 2019}\\
\textbullet{} \textit{Reliable Operating Rooms} \hfill{Northern Health, Melbourne, Nov 2018}\\
\textbullet{} \textit{Seminar on Indoor Localisation for supporting decision-making} \hfill{University of Melbourne, Nov 2018}\\
\textbullet{} \textit{PhD Confirmation Seminar - Flow Optimization based on indoor localization} \hfill{University of Melbourne, June 2018}\\


\section{Projects}
\textbf{Semantic Indoor Localisation}, \textit{PhD Research Project}\\
Supervisor: Prof. \textit{\href{https://people.eng.unimelb.edu.au/vkostakos/}{Vassilis Kostakos}}
% We are developing a system to analyse indoor localisation traces, especially through Bluetooth iBeacons and across multiple rooms. We focus more on the semantic analysis of the people's movements and less on the accuracy of the indoor localisation system. The localisation system uses an Android app developed in Java which communicates with a custom developed server written in PHP, NodeJS, HTML, CSS and Javascript through WebSockets and RESTful API. We have deployed this system in a hospital and we hope this new way of approaching indoor localisation will help environments to better analyse people's (such as patients) flow inside buildings.

\textbf{Emotion Recognition with Affectiva}, \textit{Side project at UniMelb}\\
Developed as a way of interacting with PCs through emotions and facial expressions, we used the Affectiva SDK and C\# on Windows or Java on Android.

\textbf{Injection of a scripting language into the Unreal Engine}, \textit{Master thesis}\\
Supervisor: Prof. Franco Tecchia
% We reworked a custom scripting language called XVR and initially developed by VRMedia at the Scuola Superiore Sant'Anna. We injected the language into the Unreal 3D Engine, connecting its compiler, virtual machine and debugger and adding new instructions to call Unreal's native C++ code. We also developed a VSCode plugin to enable code highlighting and live debugging through sockets. The new language showed better performance than the existing Blueprint visual scripting language, while keeping the language simple and flexible thanks to its javascript-like OOP syntax and dynamic types.


\textbf{MeshLabJS Texture and .zip support}, \textit{Final project}\\
Supervisor: Prof. Paolo Cignoni
% I helped developing the texture support for MeshlabJS including the uplad of .zip files to the system. MeshlabJS is a browser-based mesh processing tool which uses C++ and VCGLib in conjunction with Emscripten. During this project I learnt how to work with WebGL, ThreeJS and shaders, Javascript and NodeJS, Emscripten. 

\textbf{Interactive shoes showcase}, \textit{Bachelors thesis}\\
Supervisor: Prof. Riccardo Scateni
% Developed an interactive shoes showcase to highlight shoes in a store. We developed using Arduino and a desktop application developed in C++ with the Qt framework.


% \employer{LNMIIT}
% \location{}
% \title{\textbf{Next Season Scrapper\hfill Oct 2018}
%  }
% \begin{position}
% % \begin{itemize}
% A python based web application which crawls imdb.com and collects information about the upcoming episode for a given TV-Series and emailing this information to the provided email id.  
% % \end{itemize}
% \begin{itemize}
% \item \textbf{Technology/Tools:} Python, BeautifulSoup, MySQL, Omdb API
% \end{itemize}
% \begin{itemize}
% \item \textbf{Link :} github.com/Aneesh540/NextSeason
% \end{itemize}
% \end{position}


\section{Languages}
\par

\textbullet{} \textbf{Italian}, Native speaker \\
\textbullet{} \textbf{English}, Fluent, TOEFL: 112 (\textbf{C2}) \\
\textbullet{} \textbf{Spanish}, Intermediate\\

\section{Certificates}
\par
\textbullet{} \textbf{MTC-GR} - Melbourne Teaching Certificate for Graduate Research\hfill{Nov 2019}\\
\textbullet{} \textbf{WWCC} - Working With Children Certificate\hfill{Feb 2019}\\
\textbullet{} \textit{UniMelb Tutor Development Program}\hfill{Jun 2018}\\
\textbullet{} \textbf{RIOT} - Research Integrity Online Training\hfill{Jun 2018}\\
\textbullet{} \textbf{TOEFL} 112/120 (Reading: 28, Listening: 28, Speaking: 26, Writing: 30)\hfill{Sep 2017}\\
\textbullet{} \textbf{PADI} - Open Water Diving License

\section{Awards}
\textbullet{} \textbf{PhD Scholarship} awarded by \textit{CSIRO - Data61}\\
\textbullet{} \textbf{Top graduates award} awarded by \textit{Universit\'{a} di Pisa}\\
\textbullet{} \textbf{Top graduates award} awarded by \textit{ERSU Cagliari}

% \normalfont\sectionfont

\section{Relevant\\courses}
\par
\textbullet{} Designing Novel Interaction (Distinguished Project Award) \\
\textbullet{} Research Methods
\textbullet{} Computer Graphics
\textbullet{} Linear Algebra \\
\textbullet{} Introduction to Robotics
\textbullet{} Virtual Environments \\
\textbullet{} Programming Languages and Compilers
\textbullet{} Advanced Databases \\
\textbullet{} Advanced Programming
\textbullet{} Advanced Algorithms

\section{Additional Activities}

\textbullet{} Attended CHI 2021 remotely\\
\textbullet{} SV (Student Volunteer) at UbiComp 2020 Online\\
\textbullet{} Doctoral Colloquium and Workshop presenter at UbiComp 2019 in London, UK\\
\textbullet{} SV (Student Volunteer) at UbiComp 2019 in London, UK\\
\textbullet{} Open Day Volunteer at the University of Melbourne, Australia \\
\textbullet{} SV (Student Volunteer) and attendee at OzCHI 2018 in Melbourne, Australia \\
\textbullet{} Attended UbiComp 2018 in Singapore\\


\section{Extras}

\textbullet{} Amateur photographer (landscape, astro-photography and aerial)\\
\textbullet{} Traveller and backpacker\\
\textbullet{} Long distance and open water swimmer\\


\section{Publications}
%\nocite{*}
%\bibliographystyle{ACM-Reference-Format}
%\bibliography{publications}     
\begin{enumerate}[leftmargin=0pt]
    \item  \underline{G. Marini}, B. Tag, J. Goncalves, E. Velloso, R. Jurdak, V. Kostakos. Measuring Mobility and Room Occupancy in Clinical Settings: System Development and Implementation,DOI: 10.2196/19874
    \item  \underline{G. Marini}. Towards indoor localisation analytics for modelling flows of movements. 2019. International Joint Conference on Pervasive and Ubiquitous Computing Adjunct (UbiComp '19 Adjunct) 377–382. doi:10.1145/3341162.3349306
     \item \underline{G. Marini}, J. Gonçalves, E. Velloso, R. Jurdak, and V. Kostakos. 2019. Towards context-free semantic localisation. International Joint Conference on Pervasive and Ubiquitous Computing Adjunct (UbiComp '19 Adjunct), 584–591.\\ doi: 10.1145/3341162.3349329.
     \item Northern Health gets its teeth into patient flow. 2019. VicHealth  [\href{http://www.health.vic.gov.au/healthvictoria/jul19/flow.htm}{URL}]
    \item W. Jiang, \underline{G. Marini}, N. Berkel, Z. Sarsenbayeva, Z. Tan, C. Luo, X. He, T. Dingler, J. Goncalves, Y. Kawahara, V. Kostakos. 2019. Probing Sucrose Contents in Everyday Drinks Using Miniaturized Near-Infrared Spectroscopy Scanners. Proceedings of the ACM on Interactive, Mobile, Wearable and Ubiquitous Technologies (IMWUT), 3(4), 136:1–136:25. doi: 10.1145/3369834. [ERA: A*].
    \item  W. Jiang, \underline{G. Marini}, N. Berkel, Z. Sarsenbayeva, C. Luo, X. He, T. Dingler, Y. Kawahara, V. Kostakos. 2018. A Mobile Scanner for Probing Liquid Samples in Everyday Settings. 19 International Joint Conference on Pervasive and Ubiquitous Computing Adjunct (UbiComp Adjunct). ACM, 1172–1177. doi: 10.1145/3267305.3274764.
    \item Z. Sarsenbayeva, \underline{G. Marini}, N. van Berkel, C. Luo, W. Jiang, K. Yang, G. Wadley, T. Dingler, V. Kostakos, J. Goncalves, "Does Smartphone Use Drive our Emotions or vice versa? A Causal Analysis", in Proceedings of ACM SIGCHI Conference on Human Factors in Computing Systems (CHI’20), 2020, to appear.   
    \item D. Hettiachchi, Z. Sarsenbayeva, F. Allison, N. van Berkel, T. Dingler, \underline{G. Marini}, V. Kostakos, J. Goncalves, "“Hi! I am the Crowd Tasker” – Crowdsourcing through Digital Voice Assistants", in Proceedings of ACM SIGCHI Conference on Human Factors in Computing Systems (CHI’20), 2020, to appear.   

    
\end{enumerate}

%-------------------------------------------------------------------------------


\end{resume}
\(\)\end{document}